% Options for packages loaded elsewhere
\PassOptionsToPackage{unicode}{hyperref}
\PassOptionsToPackage{hyphens}{url}
\PassOptionsToPackage{dvipsnames,svgnames*,x11names*}{xcolor}
%
\documentclass[
]{report}
\usepackage{lmodern}
\usepackage{amssymb,amsmath}
\usepackage{ifxetex,ifluatex}
\ifnum 0\ifxetex 1\fi\ifluatex 1\fi=0 % if pdftex
  \usepackage[T1]{fontenc}
  \usepackage[utf8]{inputenc}
  \usepackage{textcomp} % provide euro and other symbols
\else % if luatex or xetex
  \usepackage{unicode-math}
  \defaultfontfeatures{Scale=MatchLowercase}
  \defaultfontfeatures[\rmfamily]{Ligatures=TeX,Scale=1}
\fi
% Use upquote if available, for straight quotes in verbatim environments
\IfFileExists{upquote.sty}{\usepackage{upquote}}{}
\IfFileExists{microtype.sty}{% use microtype if available
  \usepackage[]{microtype}
  \UseMicrotypeSet[protrusion]{basicmath} % disable protrusion for tt fonts
}{}
\makeatletter
\@ifundefined{KOMAClassName}{% if non-KOMA class
  \IfFileExists{parskip.sty}{%
    \usepackage{parskip}
  }{% else
    \setlength{\parindent}{0pt}
    \setlength{\parskip}{6pt plus 2pt minus 1pt}}
}{% if KOMA class
  \KOMAoptions{parskip=half}}
\makeatother
\usepackage{xcolor}
\IfFileExists{xurl.sty}{\usepackage{xurl}}{} % add URL line breaks if available
\IfFileExists{bookmark.sty}{\usepackage{bookmark}}{\usepackage{hyperref}}
\hypersetup{
  pdftitle={Ovarian Cancer Histotypes: Report of Statistical Findings},
  pdfauthor={Derek Chiu},
  colorlinks=true,
  linkcolor=Maroon,
  filecolor=Maroon,
  citecolor=Blue,
  urlcolor=Blue,
  pdfcreator={LaTeX via pandoc}}
\urlstyle{same} % disable monospaced font for URLs
\usepackage[margin=1in]{geometry}
\usepackage{longtable,booktabs}
% Correct order of tables after \paragraph or \subparagraph
\usepackage{etoolbox}
\makeatletter
\patchcmd\longtable{\par}{\if@noskipsec\mbox{}\fi\par}{}{}
\makeatother
% Allow footnotes in longtable head/foot
\IfFileExists{footnotehyper.sty}{\usepackage{footnotehyper}}{\usepackage{footnote}}
\makesavenoteenv{longtable}
\usepackage{graphicx}
\makeatletter
\def\maxwidth{\ifdim\Gin@nat@width>\linewidth\linewidth\else\Gin@nat@width\fi}
\def\maxheight{\ifdim\Gin@nat@height>\textheight\textheight\else\Gin@nat@height\fi}
\makeatother
% Scale images if necessary, so that they will not overflow the page
% margins by default, and it is still possible to overwrite the defaults
% using explicit options in \includegraphics[width, height, ...]{}
\setkeys{Gin}{width=\maxwidth,height=\maxheight,keepaspectratio}
% Set default figure placement to htbp
\makeatletter
\def\fps@figure{htbp}
\makeatother
\setlength{\emergencystretch}{3em} % prevent overfull lines
\providecommand{\tightlist}{%
  \setlength{\itemsep}{0pt}\setlength{\parskip}{0pt}}
\setcounter{secnumdepth}{5}
\usepackage{titlesec, blindtext, color, float}

\titleformat{\chapter}[display]
  {\Huge\bfseries}
  {}
  {0pt}
  {\thechapter.\ }

\titleformat{name=\chapter,numberless}[display]
  {\Huge\bfseries}
  {}
  {0pt}
  {}

\titlespacing*{\chapter}{0pt}{0pt}{40pt}

\renewcommand{\bibname}{References}
\usepackage[]{natbib}
\bibliographystyle{apalike}

\title{Ovarian Cancer Histotypes: Report of Statistical Findings}
\author{Derek Chiu}
\date{2020-11-06}

\begin{document}
\maketitle

{
\hypersetup{linkcolor=}
\setcounter{tocdepth}{1}
\tableofcontents
}
\listoftables
\listoffigures
\hypertarget{preface}{%
\chapter*{Preface}\label{preface}}
\addcontentsline{toc}{chapter}{Preface}

This report of statistical findings describes the classification of ovarian cancer histotypes using data from NanoString CodeSets.

Marina Pavanello conducted the initial exploratory data analysis, Cathy Tang implemented class imbalance techniques, Derek Chiu conducted the normalization and statistical analysis, and Aline Talhouk lead the project.

\hypertarget{introduction}{%
\chapter{Introduction}\label{introduction}}

Ovarian cancer has five major histotypes: high-grade serous carcinoma (HGSC), low-grade serous carcinoma (LGSC), endometrioid carcinoma (ENOC), mucinous carcinoma (MUC), and clear cell carcinoma (CCOC). A common problem with classifying these histotypes is that there is a class imbalance issue. HGSC dominates the distribution, commonly accounting for 70\% of cases in many patient cohorts, while the other four histotypes are spread over the rest of the cases.

In the NanoString CodeSets, we also run into a problem with trying to find suitable control pools to normalize the gene expression. For prospective NanoString runs, the pools can be specifically chosen, but for retrospective runs, we have to utilize a combination of common samples and common genes as references for normalization.

The supervised learning is performed under a consensus framework: we consider various classification algorithms and use evaluation metrics to help make decisions of which methods to carry forward for downstream analysis.

\hypertarget{methods}{%
\chapter{Methods}\label{methods}}

\hypertarget{data-processing}{%
\section{Data Processing}\label{data-processing}}

RNA was extracted from FFPE ovarian carcinoma samples and expression was quantified using NanoString nCounter. Samples were run in three CodeSets. Some samples or pools of samples were repeated across CodeSets for expression normalization. Normalizing CS2 to CS3 can easily follow the \href{https://dchiu911.shinyapps.io/PrOType/}{PrOType} method for HGSC subtypes because both CodeSets have pool samples. A different technique is implemented when normalizing across CS1, CS2, and CS3 where we use common samples and genes as reference sets.

\hypertarget{raw-data}{%
\subsection{Raw Data}\label{raw-data}}

NanoString CodeSets contained a mix of all probes of interest, six positive controls spiked-in at fixed proportional concentrations (0.125- 128 fM), and eight negative controls (probes without a corresponding target). Gene targets also included 5 housekeeping genes: POLR1B, SDHA, PGK1, ACTB, RPL19. Gene selection was made from top ranked differential gene expression analysis between ovarian cancer histotypes and molecular subtypes of HGSC, as well as containing some genes of interest from unrelated projects. Gene targets in each subsequent CodeSet were re-curated, where non-informative genes were dropped and new potential differentiating genes were added.

There are 3 NanoString CodeSets:

\begin{itemize}
\item
  CS1: OvCa2103\_C953

  \begin{itemize}
  \tightlist
  \item
    Samples = 412
  \item
    Genes = 275
  \end{itemize}
\item
  CS2: PrOTYPE2\_v2\_C1645

  \begin{itemize}
  \tightlist
  \item
    Samples = 1223
  \item
    Genes = 384
  \end{itemize}
\item
  CS3: OTTA2014\_C2822

  \begin{itemize}
  \tightlist
  \item
    Samples = 5424
  \item
    Genes = 532
  \end{itemize}
\end{itemize}

These datasets contain raw counts extracted straight from NanoString RCC files.

\hypertarget{housekeeping-genes}{%
\subsection{Housekeeping Genes}\label{housekeeping-genes}}

The first normalization step is to normalize all endogenous genes to housekeeping genes (POLR1B, SDHA, PGK1, ACTB, RPL19; reference genes expressed in all cells). We normalize by subtracting the average log\textsubscript{2} housekeeping gene expression from the log\textsubscript{2} endogenous gene expression:

log\textsubscript{2} endogenous expression - log\textsubscript{2} average housekeeping expression = relative expression

The updated CodeSet dimensions are now:

\begin{itemize}
\item
  CS1: OvCa2103\_C953

  \begin{itemize}
  \tightlist
  \item
    Samples = 412
  \item
    Genes = 256
  \end{itemize}
\item
  CS2: PrOTYPE2\_v2\_C1645

  \begin{itemize}
  \tightlist
  \item
    Samples = 1223
  \item
    Genes = 365
  \end{itemize}
\item
  CS3: OTTA2014\_C2822

  \begin{itemize}
  \tightlist
  \item
    Samples = 5424
  \item
    Genes = 513
  \end{itemize}
\end{itemize}

The number of genes are reduced by 19: 5 housekeeping, 8 negative, 6 positive (the latter 2 types are not used).

\hypertarget{common-samples-and-genes}{%
\subsection{Common Samples and Genes}\label{common-samples-and-genes}}

Since the reference pool samples only exist in CS2 and CS3, we need to find an alternative method to normalize all three CodeSets. One method is to select common samples and common genes that exist in all three. We found 72 common genes. Using the \texttt{summaryID} identifier, we also found 78 common summary IDs, translating to 320 samples. The number of samples that were matched to each CodeSet differed:

\begin{itemize}
\item
  CS1: OvCa2103\_C953

  \begin{itemize}
  \tightlist
  \item
    Samples = 93
  \item
    Genes = 72
  \end{itemize}
\item
  CS2: PrOTYPE2\_v2\_C1645

  \begin{itemize}
  \tightlist
  \item
    Samples = 87
  \item
    Genes = 72
  \end{itemize}
\item
  CS3: OTTA2014\_C2822

  \begin{itemize}
  \tightlist
  \item
    Samples = 140
  \item
    Genes = 72
  \end{itemize}
\end{itemize}

\hypertarget{overlap-of-common-samples-by-summary-id}{%
\subsubsection{Overlap of common samples by summary ID}\label{overlap-of-common-samples-by-summary-id}}

\begin{center}\includegraphics{OV_Histotypes_RSF_files/figure-latex/venn-id-1} \end{center}

\hypertarget{overlap-of-common-genes}{%
\subsubsection{Overlap of common genes}\label{overlap-of-common-genes}}

\begin{center}\includegraphics{OV_Histotypes_RSF_files/figure-latex/venn-genes-1} \end{center}

*Excluding housekeeping genes and controls

\hypertarget{cs1-training-set-generation}{%
\subsection{CS1 Training Set Generation}\label{cs1-training-set-generation}}

We use the reference method to normalize CS1 to CS3.

\begin{itemize}
\item
  CS1 reference set: duplicate samples from CS1

  \begin{itemize}
  \tightlist
  \item
    Samples = 25
  \item
    Genes = 72
  \end{itemize}
\item
  CS3 reference set: corresponding samples in CS3 also found in CS1 reference set

  \begin{itemize}
  \tightlist
  \item
    Samples = 20
  \item
    Genes = 72
  \end{itemize}
\item
  CS1 validation set: remaining CS1 samples with reference set removed

  \begin{itemize}
  \tightlist
  \item
    Samples = 387
  \item
    Genes = 72
  \end{itemize}
\end{itemize}

The final CS1 training set has 304 samples on 72 genes after normalization and keeping only the major histotypes of interest.

\hypertarget{cs2-training-set-generation}{%
\subsection{CS2 Training Set Generation}\label{cs2-training-set-generation}}

We use the pool method to normalize CS2 to CS3 so we can be consistent with the PrOType normalization when there are available pools.

\begin{itemize}
\item
  CS2 pools:

  \begin{itemize}
  \tightlist
  \item
    Samples = 12 (Pool 1 = 4, Pool 2 = 4, Pool 3 = 4)
  \item
    Genes = 365
  \end{itemize}
\item
  CS3 pools:

  \begin{itemize}
  \tightlist
  \item
    Samples = 22 (Pool 1 = 12, Pool 2 = 5, Pool 3 = 5)
  \item
    Genes = 513
  \end{itemize}
\item
  CS2 validation set: CS2 samples with pools removed

  \begin{itemize}
  \tightlist
  \item
    Samples = 1214
  \item
    Genes = 365
  \end{itemize}
\end{itemize}

The final CS2 training set has 945 samples on 136 (common) genes after normalization and keeping only the major histotypes of interest.

\hypertarget{cohort-distribution}{%
\subsection{Cohort Distribution}\label{cohort-distribution}}

CodeSets comprised samples from sites collected internationally as shown below. Note that the CS3 pools sample total (n=58) shown here include those that are not used as reference pools, following previous normalization methods. In particular, the distribution of CS3 pools actually used for normalization (n=22) is POOL1 = 12, POOL2 = 5, POOL3 = 5.

\begin{table}

\caption{\label{tab:cohort-dist}Cohort Distribution amongst CodeSets}
\centering
\begin{tabular}[t]{l|r|r|r}
\hline
cohort & cs1 & cs2 & cs3\\
\hline
MAYO & 6 & 63 & NA\\
\hline
MTL & 3 & 59 & NA\\
\hline
OOU & 108 & 43 & 19\\
\hline
OOUE & 32 & 30 & 11\\
\hline
VOA & 145 & 122 & 538\\
\hline
ICON7 & NA & 416 & NA\\
\hline
JAPAN & NA & 8 & NA\\
\hline
OVAR3 & NA & 150 & NA\\
\hline
POOL-CTRL & NA & 12 & NA\\
\hline
DOVE4 & NA & NA & 1160\\
\hline
POOL-1 & NA & NA & 31\\
\hline
POOL-2 & NA & NA & 14\\
\hline
POOL-3 & NA & NA & 13\\
\hline
TNCO & NA & NA & 691\\
\hline
\end{tabular}
\end{table}

\hypertarget{normalization-between-codesets}{%
\section{Normalization Between CodeSets}\label{normalization-between-codesets}}

After normalization to housekeeping genes and filtering for the five major histotypes of interest, as determined by pathology review and/or IHC, two methods were used to normalize data between CodeSets.

\hypertarget{common-samples-method}{%
\subsection{Common Samples Method}\label{common-samples-method}}

The common samples method was used to normalize CodeSet1, 2, and 3, where common samples and genes were used as reference sets. Among the samples repeated in all CodeSets we normalized using either: a random set of 3 samples from each major histotype (random3; n=15), a random set of 2 samples from each major histotype (random2; n=10), or a random set of 1 sample from each major histotype (random1; n=5). In each case CodeSet3 expression (X\textsubscript{3}) was held fixed, while CodeSet1/2 expression (X\textsubscript{1} and X\textsubscript{2}) were normalized to CodeSet3 by subtracting the average gene expression from the CodeSet1/2 reference set (R\textsubscript{1} or R\textsubscript{2}) and adding the average gene expression of the CodeSet3 reference set (R\textsubscript{3}). Alternatively, X\textsubscript{1} (norm) = X\textsubscript{1} - R\textsubscript{1} + R\textsubscript{3} would calibrate CodeSet1 to CodeSet3.

\hypertarget{pools-method}{%
\subsection{Pools Method}\label{pools-method}}

The pools method was used to normalize CodeSet2 and CodeSet3. The three reference pools, regularly assayed mixes of samples representing all histotypes, were run in CodeSet2 and CodeSet3 only. CodeSet2 contained 12 reference pool samples (Pool 1 = 4, Pool 2 = 4, Pool 3 = 4) and CodeSet3 contained 22 reference pool samples (Pool 1 = 12, Pool 2 = 5, Pool 3 = 5). Similar to the common samples method, CodeSet2 was normalized to CodeSet3 via: X\textsubscript{2} (norm) = X\textsubscript{2} - R\textsubscript{2} + R\textsubscript{3} where R is the average expression of the reference pool samples in the respective CodeSet. This method of pool normalization was also used by PrOType to classify HGSC subtypes

\hypertarget{concordance-comparison}{%
\subsection{Concordance Comparison}\label{concordance-comparison}}

Concordance between CodeSets using the different normalization strategies was compared in common samples, excluding those used for the normalization, using Pearson's correlation coefficient (R\textsuperscript{2}), coefficient of accuracy (Ca), and Lin's concordance correlation (Rc = R\textsuperscript{2} x Ca).

\hypertarget{histotype-classification}{%
\section{Histotype Classification}\label{histotype-classification}}

We use 6 classification algorithms and 4 subsampling methods across 500 repetitions in the supervised learning framework for CS1 and CS2. The pipeline was run using many SGE batch jobs as a way of parallelization on a CentOS 5 server. Implementations of the techniques below were called from the \href{https://alinetalhouk.github.io/splendid/}{splendid} package.

\begin{itemize}
\item
  Classifiers:

  \begin{itemize}
  \tightlist
  \item
    Random Forest
  \item
    Adaboost
  \item
    XGBoost
  \item
    LDA
  \item
    SVM
  \item
    K-Nearest Neighbours
  \end{itemize}
\item
  Subsampling:

  \begin{itemize}
  \tightlist
  \item
    None
  \item
    Down-sampling
  \item
    Up-sampling
  \item
    SMOTE
  \end{itemize}
\end{itemize}

\hypertarget{validation}{%
\chapter{Validation}\label{validation}}

\hypertarget{full-data-distributions}{%
\section{Full Data Distributions}\label{full-data-distributions}}

The histotype distributions on the full data are shown below.

\begin{table}

\caption{\label{tab:dist-all-gr}All CodeSet Histotype Groups}
\centering
\begin{tabular}[t]{l|r|r|r}
\hline
hist\_gr & CS1 & CS2 & CS3\\
\hline
HGSC & 169 & 757 & 2453\\
\hline
non-HGSC & 196 & 377 & 677\\
\hline
\end{tabular}
\end{table}

\begin{table}

\caption{\label{tab:dist-all}All CodeSet Histotypes}
\centering
\begin{tabular}[t]{l|r|r|r}
\hline
revHist & CS1 & CS2 & CS3\\
\hline
CARCINOMA-NOS & 0 & 61 & 23\\
\hline
Carcinoma, NOS & 0 & 0 & 2\\
\hline
CCOC & 57 & 68 & 182\\
\hline
CCOC-MCT & 0 & 1 & 0\\
\hline
Cell-Line & 17 & 48 & 13\\
\hline
CTRL & 0 & 12 & 0\\
\hline
ENOC & 61 & 30 & 272\\
\hline
ENOC-CCOC & 0 & 7 & 0\\
\hline
ERROR & 0 & 3 & 0\\
\hline
HGSC & 169 & 757 & 2453\\
\hline
HGSC-MCT & 0 & 1 & 0\\
\hline
LGSC & 22 & 29 & 50\\
\hline
MBOT & 0 & 20 & 3\\
\hline
MET-NOP & 0 & 21 & 0\\
\hline
MIXED (ENOC/CCOC) & 0 & 0 & 1\\
\hline
MIXED (ENOC/LGSC) & 0 & 0 & 1\\
\hline
MIXED (HGSC/CCOC) & 0 & 0 & 1\\
\hline
mixed cell & 0 & 0 & 7\\
\hline
MMMT & 0 & 0 & 30\\
\hline
MUC & 20 & 61 & 77\\
\hline
Other (use when 6, 7, or 9 is not distinguished) or unknown if epithelial & 0 & 0 & 1\\
\hline
Other/Exclude & 0 & 0 & 8\\
\hline
SBOT & 19 & 10 & 3\\
\hline
Serous & 0 & 0 & 2\\
\hline
serous LMP & 0 & 0 & 1\\
\hline
SQAMOUS & 0 & 1 & 0\\
\hline
UNK & 0 & 4 & 0\\
\hline
\end{tabular}
\end{table}

\begin{table}

\caption{\label{tab:dist-common}Common Summary ID CodeSet Histotypes}
\centering
\begin{tabular}[t]{l|r|r|r}
\hline
revHist & CS1 & CS2 & CS3\\
\hline
CCOC & 3 & 4 & 9\\
\hline
Cell-Line & 4 & 5 & 5\\
\hline
ENOC & 4 & 4 & 9\\
\hline
HGSC & 68 & 64 & 98\\
\hline
LGSC & 7 & 5 & 8\\
\hline
MUC & 7 & 5 & 11\\
\hline
\end{tabular}
\end{table}

\begin{table}

\caption{\label{tab:dist-major-hist}All CodeSet Major Histotypes}
\centering
\begin{tabular}[t]{l|r|r|r|r|r|r}
\hline
revHist & CS1 & CS2 & CS3 & CS1\_percent & CS2\_percent & CS3\_percent\\
\hline
CCOC & 57 & 68 & 182 & 17.3 & 7.2 & 6.0\\
\hline
ENOC & 61 & 30 & 272 & 18.5 & 3.2 & 9.0\\
\hline
HGSC & 169 & 757 & 2453 & 51.4 & 80.1 & 80.9\\
\hline
LGSC & 22 & 29 & 50 & 6.7 & 3.1 & 1.6\\
\hline
MUC & 20 & 61 & 77 & 6.1 & 6.5 & 2.5\\
\hline
\end{tabular}
\end{table}

\begin{table}

\caption{\label{tab:dist-cs1}CS1 Histotypes}
\centering
\begin{tabular}[t]{l|l|r}
\hline
CodeSet & revHist & n\\
\hline
CS1 & CCOC & 57\\
\hline
CS1 & Cell-Line & 17\\
\hline
CS1 & ENOC & 61\\
\hline
CS1 & HGSC & 169\\
\hline
CS1 & LGSC & 22\\
\hline
CS1 & MUC & 20\\
\hline
CS1 & SBOT & 19\\
\hline
\end{tabular}
\end{table}

\begin{table}

\caption{\label{tab:dist-cs2}CS2 Histotypes}
\centering
\begin{tabular}[t]{l|l|r}
\hline
CodeSet & revHist & n\\
\hline
CS2 & CARCINOMA-NOS & 61\\
\hline
CS2 & CCOC & 68\\
\hline
CS2 & CCOC-MCT & 1\\
\hline
CS2 & Cell-Line & 48\\
\hline
CS2 & CTRL & 12\\
\hline
CS2 & ENOC & 30\\
\hline
CS2 & ENOC-CCOC & 7\\
\hline
CS2 & ERROR & 3\\
\hline
CS2 & HGSC & 757\\
\hline
CS2 & HGSC-MCT & 1\\
\hline
CS2 & LGSC & 29\\
\hline
CS2 & MBOT & 20\\
\hline
CS2 & MET-NOP & 21\\
\hline
CS2 & MUC & 61\\
\hline
CS2 & SBOT & 10\\
\hline
CS2 & SQAMOUS & 1\\
\hline
CS2 & UNK & 4\\
\hline
\end{tabular}
\end{table}

\begin{table}

\caption{\label{tab:dist-cs3}CS3 Histotypes}
\centering
\begin{tabular}[t]{l|l|r}
\hline
CodeSet & revHist & n\\
\hline
CS3 & CARCINOMA-NOS & 23\\
\hline
CS3 & Carcinoma, NOS & 2\\
\hline
CS3 & CCOC & 182\\
\hline
CS3 & Cell-Line & 13\\
\hline
CS3 & ENOC & 272\\
\hline
CS3 & HGSC & 2453\\
\hline
CS3 & LGSC & 50\\
\hline
CS3 & MBOT & 3\\
\hline
CS3 & MIXED (ENOC/CCOC) & 1\\
\hline
CS3 & MIXED (ENOC/LGSC) & 1\\
\hline
CS3 & MIXED (HGSC/CCOC) & 1\\
\hline
CS3 & mixed cell & 7\\
\hline
CS3 & MMMT & 30\\
\hline
CS3 & MUC & 77\\
\hline
CS3 & Other (use when 6, 7, or 9 is not distinguished) or unknown if epithelial & 1\\
\hline
CS3 & Other/Exclude & 8\\
\hline
CS3 & SBOT & 3\\
\hline
CS3 & Serous & 2\\
\hline
CS3 & serous LMP & 1\\
\hline
\end{tabular}
\end{table}

\hypertarget{training-set-distributions}{%
\section{Training Set Distributions}\label{training-set-distributions}}

The training set distributions for CS1 and CS2 are shown below.

\begin{table}

\caption{\label{tab:training-dist-cs1}CS1 Training Set Histotypes}
\centering
\begin{tabular}[t]{l|r}
\hline
histotype & n\\
\hline
CCC & 57\\
\hline
ENOCa & 59\\
\hline
HGSC & 156\\
\hline
LGSC & 16\\
\hline
MUC & 16\\
\hline
\end{tabular}
\end{table}

\begin{table}

\caption{\label{tab:training-dist-cs2}CS2 Training Set Histotypes}
\centering
\begin{tabular}[t]{l|r}
\hline
histotype & n\\
\hline
CCOC & 68\\
\hline
ENOC & 30\\
\hline
HGSC & 757\\
\hline
LGSC & 29\\
\hline
MUC & 61\\
\hline
\end{tabular}
\end{table}

\hypertarget{normalization}{%
\section{Normalization}\label{normalization}}

\begin{center}\includegraphics{OV_Histotypes_RSF_files/figure-latex/cc-raw-1} \end{center}

\begin{center}\includegraphics{OV_Histotypes_RSF_files/figure-latex/cc-hknorm-1} \end{center}

\hypertarget{common-samples-method-1}{%
\subsection{Common Samples Method}\label{common-samples-method-1}}

We employ a new normalization technique using randomly selected samples common to all three CodeSets with a uniform distribution of histotypes as the reference dataset. The number of randomly selected samples ranges from 1-3 per histotype. Hence, the reference dataset has either 5, 10, or 15 samples and we validate on the remaining. Note that ottaID duplicates are collapsed by mean averaging the gene expression. n=72 common samples.

CodeSets 1 and 2 are calibrated to CodeSet3 as follows:\\
\texttt{X\^{}1(norm)\ =\ X\^{}1\ -\ R\^{}1\ +\ R\^{}3}~\\
\texttt{X\^{}2(norm)\ =\ X\^{}2\ -\ R\^{}2\ +\ R\^{}3}~\\
\texttt{X\^{}3(norm)\ =\ X\^{}3}

\begin{figure}[H]

{\centering \includegraphics{OV_Histotypes_RSF_files/figure-latex/cc-non3-1} 

}

\caption{Random3 Non-Normalized Concordance Measure Distributions}\label{fig:cc-non3}
\end{figure}

\begin{figure}[H]

{\centering \includegraphics{OV_Histotypes_RSF_files/figure-latex/cc-rand3-1} 

}

\caption{Random3 Normalized Concordance Measure Distributions}\label{fig:cc-rand3}
\end{figure}

\begin{figure}[H]

{\centering \includegraphics{OV_Histotypes_RSF_files/figure-latex/cc-non2-1} 

}

\caption{Random2 Non-Normalized Concordance Measure Distributions}\label{fig:cc-non2}
\end{figure}

\begin{figure}[H]

{\centering \includegraphics{OV_Histotypes_RSF_files/figure-latex/cc-rand2-1} 

}

\caption{Random2 Normalized Concordance Measure Distributions}\label{fig:cc-rand2}
\end{figure}

\begin{figure}[H]

{\centering \includegraphics{OV_Histotypes_RSF_files/figure-latex/cc-non1-1} 

}

\caption{Random1 Non-Normalized Concordance Measure Distributions}\label{fig:cc-non1}
\end{figure}

\begin{figure}[H]

{\centering \includegraphics{OV_Histotypes_RSF_files/figure-latex/cc-rand1-1} 

}

\caption{Random1 Normalized Concordance Measure Distributions}\label{fig:cc-rand1}
\end{figure}

\begin{table}

\caption{\label{tab:rand1-cs1-vs-cs3}Random1 CS1 vs. CS3 Median Concordance Measures by Histotypes}
\centering
\begin{tabular}[t]{l|r|r|r|r|r|r}
\hline
hist & R2-Non & Ca-Non & Rc-Non & R2-Norm & Ca-Norm & Rc-Norm\\
\hline
CCOC & 1.00 & 0.30 & 0.08 & 1.00 & 0.28 & 0.09\\
\hline
ENOC & 1.00 & 0.54 & 0.54 & 1.00 & 0.61 & 0.61\\
\hline
HGSC & 0.78 & 0.98 & 0.85 & 0.78 & 0.97 & 0.86\\
\hline
LGSC & 0.97 & 0.87 & 0.81 & 0.97 & 0.90 & 0.87\\
\hline
MUC & 0.75 & 0.85 & 0.68 & 0.75 & 0.82 & 0.64\\
\hline
\end{tabular}
\end{table}

\begin{table}

\caption{\label{tab:rand1-cs2-vs-cs3}Random1 CS2 vs. CS3 Median Concordance Measures by Histotypes}
\centering
\begin{tabular}[t]{l|r|r|r|r|r|r}
\hline
hist & R2-Non & Ca-Non & Rc-Non & R2-Norm & Ca-Norm & Rc-Norm\\
\hline
CCOC & 1.00 & 0.20 & 0.04 & 1.00 & 0.27 & 0.09\\
\hline
ENOC & 1.00 & 0.67 & 0.64 & 1.00 & 0.64 & 0.61\\
\hline
HGSC & 0.83 & 0.96 & 0.86 & 0.83 & 0.99 & 0.89\\
\hline
LGSC & 0.99 & 0.92 & 0.91 & 0.99 & 0.96 & 0.94\\
\hline
MUC & 0.70 & 0.78 & 0.55 & 0.70 & 0.86 & 0.57\\
\hline
\end{tabular}
\end{table}

\begin{figure}[H]

{\centering \includegraphics{OV_Histotypes_RSF_files/figure-latex/cc-rand-1} 

}

\caption{Random3 HGSC Normalized Concordance Measure Distributions}\label{fig:cc-rand}
\end{figure}

\begin{table}

\caption{\label{tab:rand3-hgsc-cs1-vs-cs3}Random3 HGSC CS1 vs. CS3 Median Concordance Measures by Histotypes}
\centering
\begin{tabular}[t]{l|r|r|r|r|r|r}
\hline
hist & R2-Non & Ca-Non & Rc-Non & R2-Norm & Ca-Norm & Rc-Norm\\
\hline
CCOC & 0.67 & 0.58 & 0.29 & 0.67 & 0.63 & 0.26\\
\hline
ENOC & 0.89 & 0.80 & 0.71 & 0.89 & 0.80 & 0.74\\
\hline
HGSC & 0.77 & 0.98 & 0.85 & 0.77 & 0.99 & 0.87\\
\hline
LGSC & 0.94 & 0.85 & 0.79 & 0.94 & 0.88 & 0.83\\
\hline
MUC & 0.76 & 0.93 & 0.73 & 0.76 & 0.93 & 0.79\\
\hline
\end{tabular}
\end{table}

\begin{table}

\caption{\label{tab:rand3-hgsc-cs2-vs-cs3}Random3 HGSC CS2 vs. CS3 Median Concordance Measures by Histotypes}
\centering
\begin{tabular}[t]{l|r|r|r|r|r|r}
\hline
hist & R2-Non & Ca-Non & Rc-Non & R2-Norm & Ca-Norm & Rc-Norm\\
\hline
CCOC & 0.69 & 0.54 & 0.35 & 0.69 & 0.64 & 0.39\\
\hline
ENOC & 0.85 & 0.77 & 0.66 & 0.85 & 0.86 & 0.76\\
\hline
HGSC & 0.82 & 0.96 & 0.86 & 0.82 & 0.99 & 0.90\\
\hline
LGSC & 0.98 & 0.95 & 0.92 & 0.98 & 0.94 & 0.92\\
\hline
MUC & 0.77 & 0.89 & 0.73 & 0.77 & 0.92 & 0.70\\
\hline
\end{tabular}
\end{table}

\hypertarget{pools-method-1}{%
\subsection{Pools Method}\label{pools-method-1}}

CodeSet2 contains 12 ref pool samples (Pool 1 = 4, Pool 2 = 4, Pool 3 = 4). CodeSet3 contains 22 ref pool samples (Pool 1 = 12, Pool 2 = 5, Pool 3 = 5). n=86 common samples.

CodeSet2 is calibrated to CodeSet3 as follows:\\
\texttt{X\^{}2(norm)\ =\ X\^{}2\ -\ R\^{}2\ +\ R\^{}3}~\\
\texttt{X\^{}3(norm)\ =\ X\^{}3}

\begin{figure}[H]

{\centering \includegraphics{OV_Histotypes_RSF_files/figure-latex/cc-cs2non-vs-cs2pools-1} 

}

\caption{CS2Non vs. CS2Pools Concordance Measure Distributions}\label{fig:cc-cs2non-vs-cs2pools}
\end{figure}

\begin{figure}[H]

{\centering \includegraphics{OV_Histotypes_RSF_files/figure-latex/cc-cs2non-vs-cs3-1} 

}

\caption{CS2 Non-Normalized Pools vs. CS3 Concordance Measure Distributions}\label{fig:cc-cs2non-vs-cs3}
\end{figure}

\begin{figure}[H]

{\centering \includegraphics{OV_Histotypes_RSF_files/figure-latex/cc-cs2pools-vs-cs3-1} 

}

\caption{CS2 Normalized Pools vs. CS3 Concordance Measure Distributions}\label{fig:cc-cs2pools-vs-cs3}
\end{figure}

\begin{table}

\caption{\label{tab:pools-cs2non-vs-cs3}Pools Non-Normalized CS2 vs. CS3 Median Concordance Measures by Histotypes}
\centering
\begin{tabular}[t]{l|r|r|r}
\hline
hist & R2 & Ca & Rc\\
\hline
CCOC & 0.66 & 0.51 & 0.29\\
\hline
ENOC & 0.87 & 0.74 & 0.64\\
\hline
HGSC & 0.77 & 0.94 & 0.80\\
\hline
LGSC & 0.98 & 0.94 & 0.92\\
\hline
MUC & 0.75 & 0.86 & 0.69\\
\hline
\end{tabular}
\end{table}

\begin{table}

\caption{\label{tab:pools-cs2norm-vs-cs3}Pools Normalized CS2 vs. CS3 Median Concordance Measures by Histotypes}
\centering
\begin{tabular}[t]{l|r|r|r}
\hline
hist & R2 & Ca & Rc\\
\hline
CCOC & 0.66 & 0.62 & 0.31\\
\hline
ENOC & 0.87 & 0.74 & 0.67\\
\hline
HGSC & 0.77 & 0.94 & 0.82\\
\hline
LGSC & 0.98 & 0.96 & 0.93\\
\hline
MUC & 0.75 & 0.91 & 0.70\\
\hline
\end{tabular}
\end{table}

\hypertarget{common-sample-distributions}{%
\section{Common Sample Distributions}\label{common-sample-distributions}}

\begin{table}

\caption{\label{tab:common-dist-all}All Common Samples Histotype Distribution}
\centering
\begin{tabular}[t]{l|r|r|r}
\hline
revHist & CS1 & CS2 & CS3\\
\hline
CCOC & 3 & 4 & 9\\
\hline
ENOC & 4 & 4 & 9\\
\hline
HGSC & 59 & 62 & 95\\
\hline
LGSC & 7 & 5 & 8\\
\hline
MUC & 7 & 5 & 11\\
\hline
\end{tabular}
\end{table}

\begin{table}

\caption{\label{tab:common-dist-distinct}Distinct Common Samples Histotype Distribution}
\centering
\begin{tabular}[t]{l|r|r|r}
\hline
revHist & CS1 & CS2 & CS3\\
\hline
CCOC & 3 & 3 & 3\\
\hline
ENOC & 3 & 3 & 3\\
\hline
HGSC & 57 & 57 & 57\\
\hline
LGSC & 4 & 4 & 4\\
\hline
MUC & 5 & 5 & 5\\
\hline
\end{tabular}
\end{table}

\hypertarget{results}{%
\chapter{Results}\label{results}}

Here we show internal validation summaries for both CS1 and CS2. The accuracy and F1-scores are the measures of interest. Algorithms are sorted by descending value. The point ranges show the median, 5th and 95th percentiles, coloured by subsampling methods.

\hypertarget{cs1}{%
\section{CS1}\label{cs1}}

\hypertarget{accuracy}{%
\subsection{Accuracy}\label{accuracy}}

\begin{figure}[H]

{\centering \includegraphics{OV_Histotypes_RSF_files/figure-latex/cs1-accuracy-1} 

}

\caption{CS1 Accuracy}\label{fig:cs1-accuracy}
\end{figure}

\hypertarget{f1-score}{%
\subsection{F1-Score}\label{f1-score}}

\begin{figure}[H]

{\centering \includegraphics{OV_Histotypes_RSF_files/figure-latex/cs1-f1-1} 

}

\caption{CS1 F1-Score}\label{fig:cs1-f1}
\end{figure}

\begin{figure}[H]

{\centering \includegraphics{OV_Histotypes_RSF_files/figure-latex/cs1-f1-class-1} 

}

\caption{CS1 Class-Specific F1-Score}\label{fig:cs1-f1-class}
\end{figure}

\hypertarget{cs2}{%
\section{CS2}\label{cs2}}

\hypertarget{accuracy-1}{%
\subsection{Accuracy}\label{accuracy-1}}

\begin{figure}[H]

{\centering \includegraphics{OV_Histotypes_RSF_files/figure-latex/cs2-accuracy-1} 

}

\caption{CS2 Accuracy}\label{fig:cs2-accuracy}
\end{figure}

\hypertarget{f1-score-1}{%
\subsection{F1-Score}\label{f1-score-1}}

\begin{figure}[H]

{\centering \includegraphics{OV_Histotypes_RSF_files/figure-latex/cs2-f1-1} 

}

\caption{CS2 F1-Score}\label{fig:cs2-f1}
\end{figure}

\begin{figure}[H]

{\centering \includegraphics{OV_Histotypes_RSF_files/figure-latex/cs2-f1-class-1} 

}

\caption{CS2 Class-Specific F1-Score}\label{fig:cs2-f1-class}
\end{figure}

  \bibliography{packages.bib}

\end{document}
